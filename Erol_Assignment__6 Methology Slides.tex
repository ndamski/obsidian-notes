\documentclass[aspectratio=169,12pt]{beamer}
\usetheme{Boadilla}
\usecolortheme{crane}
\usepackage{graphicx}

\setbeamertemplate{navigation symbols}{}
\setbeamertemplate{footline}{}

\title{Gravity Model: EU-ACP Trade Analysis}
\author{Your Name}
\institute{Your University}
\date{\today}

\begin{document}

\begin{frame}{Methodology}
  \textbf{Gravity Model via PPML}
  \begin{itemize}
    \item Theoretical basis: Anderson \& van Wincoop (2003)
     \item Industry standard for analyzing trade between countries
    \item Accounts for countries that don't trade with each other
    \item Controls for country-specific factors over time
  \end{itemize}
\end{frame}

\begin{frame}{Data}
  \textbf{Sample Size}: 78 ACP × 27 EU × 26 years = 43,238 observations
  \begin{itemize}
    \item \textbf{Trade data}: BACI database (international trade records)
    \item \textbf{Economic data}: GDP and population (World Bank)
    \item \textbf{Control variables}: Distance, language, colonial ties
  \end{itemize}
\end{frame}

\begin{frame}{Results}
  \textbf{Main Finding}: ``Stumbling block'' effect
  \begin{itemize}
    \item Countries that trade more with neighbors trade less with EU
    \item At average regional integration (10\%): 16\% less EU trade
    \item At high regional integration (26\%): 36\% less EU trade
    \item EPAs partially counteract this effect
  \end{itemize}
\end{frame}

\begin{frame}{Is the result reliable?}
  \begin{itemize}
    \item Analyzing each region separately
    \item Using different statistical methods
    \item \textit{Result: Main finding holds in all cases}
  \end{itemize}
\end{frame}

\begin{frame}{Feasibility, Weaknesses, Ethics}
  \textbf{Feasible}: Data ready, analysis complete, writing remains

  \textbf{Weaknesses}: Can't perfectly control for all factors: 
  some countries have very limited trade data
  
  \textbf{Ethical Concerns}: None, as it's public secondary data only
\end{frame}

\end{document}