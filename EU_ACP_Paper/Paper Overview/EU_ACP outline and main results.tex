\documentclass[11pt]{article}
\usepackage[utf8]{inputenc}
\usepackage{graphicx}
\usepackage{amsmath}
\usepackage{hyperref}
\usepackage{subfig}
\usepackage{booktabs}
\usepackage{makecell}
\usepackage{tabularx}

\title{EU‑ACP Outline}
\author{Noah Damski}
\date{February 22, 2026}

\begin{document}

\maketitle

\begin{abstract}
This paper examines the relationship between intra-regional trade integration among African, Caribbean, and
Pacific (ACP) countries and their bilateral trade with the European Union. Using a panel of 78 ACP countries
across seven regional economic communities from 1995 to 2020, we estimate gravity models that control for standard
trade determinants and time-varying EPA treatment. We find that higher intra-regional trade shares are associated
with significantly lower EU-ACP bilateral trade, consistent with a ``stumbling block'' effect of regional
integration. At the mean intra-regional trade share of 10 percent, bilateral trade is approximately 16 percent
lower; at the 90th percentile (26 percent), the effect reaches 36 percent. However, Economic Partnership
Agreements (EPAs) partially offset this negative relationship, suggesting that EU-ACP trade agreements can
mitigate the trade-diverting effects of regional integration. These findings have important implications for
understanding the interplay between regional integration and preferential trade arrangements in developing
countries.
\end{abstract}

\newpage

\section{Executive Summary of Results}
An examination of whether intra-regional trade integration among ACP (African,
Caribbean, Pacific) countries helps or hinders bilateral trade with EU member
states, and whether in‑force Economic Partnership Agreements (EPAs) change this
relationship.

\section{Key Findings}
\begin{enumerate}
    \item \textbf{Main Result:} Intra‑REC trade share has a
          \textbf{negative and significant} effect on EU‑ACP bilateral trade
          (coefficient~$= -1.76$, $p<0.01$). This supports the``stumbling block'' hypothesis—stronger regional
integration among
          ACP countries actually reduces their bilateral trade with the EU.

    \item \textbf{EPA Interaction:} The positive and significant interaction
          term ($0.90$, $p<0.05$) suggests that EPAs partially mitigate the
          negative effect of regional integration on trade.

    \item \textbf{Trade Direction:} The effect is slightly stronger for EU
          exports to ACP ($-1.86$) than for ACP exports to EU ($-1.60$).

    \item \textbf{Heterogeneity:} All RECs show negative effects, but with
          substantial variation—PIF (Pacific Islands) shows the largest
          negative coefficient ($-29.6$), while Central Africa shows the
          smallest effect ($-0.81$, not significant).
\end{enumerate}

\newpage

\section*{OUTLINE}

\begin{enumerate}
    \item \textbf{INTRODUCTION}
    \begin{itemize}
        \item Research question and motivation
        \item Preview of main findings
        \item Contribution to literature
    \end{itemize}

    \item \textbf{INSTITUTIONAL BACKGROUND}
    \begin{itemize}
        \item Cotonou Agreement and EPA negotiations
        \item ACP regional groupings (ECOWAS, SADC, EAC, etc.)
        \item Timeline of EPA implementation
    \end{itemize}

    \item \textbf{DATA AND DESCRIPTIVE STATISTICS}
    \begin{itemize}
        \item Data sources (BACI, CEPII, WDI)
        \item Sample construction
        \item Key variable definitions
        \item Summary statistics (Table 1)
    \end{itemize}

    \item \textbf{EMPIRICAL STRATEGY}
    \begin{itemize}
        \item Gravity model specification
        \item Fixed effects rationale
        \item Identification discussion
    \end{itemize}

    \item \textbf{MAIN RESULTS}
    \begin{itemize}
        \item Baseline gravity (Model 1)
        \item Main specification with IT Share (Model 2)
        \item EPA interaction (Model 3)
        \item Interpretation of magnitudes
    \end{itemize}

    \item \textbf{MECHANISMS AND HETEROGENEITY}
    \begin{itemize}
        \item Trade direction analysis
        \item REC-by-REC subsamples
        \item Regional subsamples (Africa/Caribbean/Pacific)
    \end{itemize}

    \item \textbf{ROBUSTNESS CHECKS}
    \begin{itemize}
        \item Sample sensitivity
        \item Specification alternatives
        \item OLS vs PPML comparison
    \end{itemize}

    \item \textbf{CONCLUSION}
    \begin{itemize}
        \item \underline{Summary of findings}:
        \begin{itemize}
            \item Strong ``stumbling block'' effect exists
            \item Higher intra‑regional trade $\rightarrow$ lower EU‑ACP bilateral trade
            \item At mean (10\%): ~16\% lower trade
            \item At 90th percentile (26\%): ~36\% lower trade
            \item EPAs partially offset the negative relationship
        \end{itemize}

        \item \underline{Policy implications}:
        \begin{itemize}
            \item Trade agreements must account for regional integration effects
            \item EPAs can mitigate trade‑diverting effects of RECs
            \item Important for development policy in developing countries
        \end{itemize}

        \item \underline{Limitations and future research}:
        \begin{itemize}
            \item Data constraints for smaller ACP countries
            \item Need deeper mechanism exploration
            \item Could extend to other developing regions
        \end{itemize}
    \end{itemize}
\end{enumerate}

\newpage


\begin{table}[t]
  \centering
  \caption{Intra-REC Trade Integration and EU-ACP Bilateral Trade}
  \label{tab:main_results}
  \vspace{-0.5em}
  \begin{scriptsize}
  \begin{tabularx}{\textwidth}{lccccc}
    \toprule
    & \makecell{Baseline\\PPML} & \makecell{IT Share\\PPML} & \makecell{IT $\times$ EPA\\PPML} &
\makecell{IT\\Intensity} & \makecell{IT Share\\OLS} \\
    \midrule
    \emph{Variables} \\
    ln(Distance)     & -1.717$^{***}$ & -1.769$^{***}$ & -1.769$^{***}$ & -1.764$^{***}$ & -1.195$^{***}$ \\
                      & (0.607)        & (0.639)        & (0.639)        & (0.639)        & (0.240) \\
    Common Language  & 1.113$^{***}$  & 1.119$^{***}$ & 1.119$^{***}$ & 1.120$^{***}$ & 0.687$^{***}$ \\
                      & (0.218)        & (0.222)        & (0.222)        & (0.222)        & (0.125) \\
    Colonial Tie     & 0.205          & 0.186          & 0.186          & 0.184          & 1.251$^{***}$ \\
                      & (0.228)        & (0.232)        & (0.232)        & (0.232)        & (0.202) \\
    EPA (=1 in force) & -0.031         & -0.058         & -0.158$^{**}$  & -0.055         & -0.136$^{***}$ \\
                      & (0.047)        & (0.044)        & (0.066)        & (0.044)        & (0.044) \\
    Intra-REC Trade Share &        & -1.755$^{***}$ & -1.758$^{***}$ &                & -1.526$^{***}$ \\
                      &                & (0.384)        & (0.378)        &                & (0.245) \\
    EPA $\times$ IT Share &           &                & 0.901$^{**}$   &                &                \\
                      &                &                & (0.422)        &                &                \\
    Intra-REC Trade Intensity &       &                &                & -0.006$^{***}$ &                \\
                      &                &                &                & (0.002)        &                \\
    \midrule
    \emph{Fixed Effects} \\
    Exporter $\times$ Year & Yes   & Yes            & Yes            & Yes            & Yes \\
    ACP Importer          & Yes   & Yes            & Yes            & Yes            & Yes \\
    Year                  & Yes   & Yes            & Yes            & Yes            & Yes \\
    \midrule
    Observations          & 44,054 & 43,238        & 43,238         & 43,238         & 40,334 \\
    R$^2$                 &        &                &                &                & 0.755 \\
    \bottomrule
  \end{tabularx}
  \end{scriptsize}

  \smallskip
  \begin{tabularx}{\textwidth}{p{15cm}}
    \emph{Notes:} Clustered (pair\_id) standard errors in parentheses. $^{***}$ $p<0.01$, $^{**}$ $p<0.05$. \\
    Columns (1)–(4) estimated via PPML. Column (5) is OLS on logged trade (zeros dropped). \\
    At the mean IT Share (10\%), column (2) implies a 16\% reduction in bilateral trade; at the 90th percentile
(26\%), the effect is a 36\% reduction. The EPA interaction in column (3) suggests EPAs partially offset the
negative regional integration effect.
  \end{tabularx}
\end{table}

\newpage

\begin{figure}[htbp]
    \centering
    \subfloat[Regional Trends]{%
        \includegraphics[width=0.45\textwidth]{it_share_time_series.png}%
        \label{fig:rec_trends}
}
    \subfloat[IT Share vs Trade]{%
        \includegraphics[width=0.45\textwidth]{stumbling_block_scatter.png}%
        \label{fig:scatter}
}
    \caption{Regional Integration and EU-ACP Trade: Descriptive Evidence}
    \label{fig:descriptive}
\end{figure}

\begin{figure}[htbp]
    \centering
    \subfloat[REC Heterogeneity]{%
        \includegraphics[width=0.48\textwidth]{rec_coef_plot.png}%
}
    \subfloat[Marginal Effects]{%
        \includegraphics[width=0.48\textwidth]{marginal_effects_it_share.png}%
}
    \caption{Effect of Intra-REC Trade Share on EU-ACP Bilateral Trade}
    \label{fig:main_results}
\end{figure}

\end{document}