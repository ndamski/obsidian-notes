\documentclass[aspectratio=169,12pt]{beamer}

% === PACKAGES ===
\usepackage[utf8]{inputenc}
\usepackage{graphicx}
\usepackage{amsmath}
\usepackage{hyperref}
\usepackage{booktabs}           % for professional tables
\usepackage{colortbl}           % for colored tables
\usepackage{xcolor}
\usepackage{subfig}
\usepackage{geometry}
\usepackage{makecell}

% === COLOR THEME ===
\definecolor{uva}{RGB}{0,51,141}     % University blue
\definecolor{uvaLight}{RGB}{100,149,237}
\definecolor{darkgray}{RGB}{64,64,64}

\setbeamercolor{title}{fg=uva}
\setbeamercolor{subtitle}{fg=darkgray}
\setbeamercolor{structure}{fg=uva}
\setbeamercolor{alerted text}{fg=red}
\setbeamercolor{item}{fg=uva}

% === BEAMER THEME SETTINGS ===
\usetheme{Madrid}
\usecolortheme{whale}
\setbeamertemplate{navigation symbols}{}
\setbeamertemplate{footline}[frame number]
\setbeamertemplate{itemize items}[circle]
\setbeamertemplate{enumerate items}[circle]

% === TITLE SLIDE INFORMATION ===
\title[EU-ACP Trade Integration]{Intra-Regional Trade Integration and EU-ACP Bilateral Trade}
\subtitle{Evidence from Gravity Model Estimations}
\author[Noah Damski]{Noah Damski}
\institute[University of Florida]{Department of Economics \\ University of Florida}
\date{February 22, 2026}

% === BEGIN DOCUMENT ===
\begin{document}

% ============================================
% TITLE SLIDE
% ============================================
\begin{frame}
    \titlepage
\end{frame}

% ============================================
% OUTLINE SLIDE
% ============================================
\begin{frame}{Presentation Outline}
    \begin{enumerate}
        \item Introduction \& Motivation
        \item Institutional Background
        \item Data \& Descriptive Statistics
        \item Empirical Strategy
        \item Main Results
        \item Mechanisms \& Heterogeneity
        \item Robustness Checks
        \item Conclusion
    \end{enumerate}
\end{frame}

% ============================================
% INTRODUCTION
% ============================================
\section{INTRODUCTION}
\begin{frame}{Introduction}
    \textbf{Research Question}
    \begin{itemize}
        \item How does intra-regional trade integration among ACP countries affect their bilateral trade with the
EU?
    \end{itemize}

    \vspace{0.3cm}

    \textbf{Key Findings}
    \begin{itemize}
        \item Higher intra-regional trade shares are associated with significantly lower EU-ACP bilateral trade
        \item ``Stumbling block'' effect: At mean IT share (10$\%$), bilateral trade is $\approx$16$\%$ lower
        \item At 90th percentile (26$\%$), the effect reaches 36$\%$ lower
        \item EPAs partially offset this negative relationship
    \end{itemize}

    \vspace{0.3cm}

    \textbf{Contribution}
    \begin{itemize}
        \item First paper to empirically test the stumbling block hypothesis in EU-ACP context
        \item Policy relevance for trade agreement design in developing countries
    \end{itemize}
\end{frame}

% ============================================
% INSTITUTIONAL BACKGROUND
% ============================================
\section{INSTITUTIONAL BACKGROUND}
\begin{frame}{Institutional Background}
    \textbf{Cotonou Agreement \& EPA Negotiations}
    \begin{itemize}
        \item Signed in 2000, replaced Lomé Convention
        \item Key change: From non-reciprocal to reciprocal trade agreements
        \item Economic Partnership Agreements (EPAs) as new framework
    \end{itemize}

    \vspace{0.3cm}

    \textbf{ACP Regional Groupings}
    \begin{itemize}
        \item ECOWAS (West Africa), SADC (Southern Africa)
        \item EAC (East Africa), CARICOM (Caribbean)
        \item CEMAC, COMESA, Pacific Islands Forum
    \end{itemize}

    \vspace{0.3cm}

    \textbf{Timeline}
    \begin{itemize}
        \item 2000: Cotonou Agreement signed
        \item 2002-2007: EPA negotiations begin
        \item 2008 onwards: EPA implementation phases
    \end{itemize}
\end{frame}

% ============================================
% DATA & DESCRIPTIVE STATISTICS
% ============================================
\section{DATA AND DESCRIPTIVE STATISTICS}
\begin{frame}{Data Sources}
    \begin{columns}
        \begin{column}{0.5\textwidth}
            \textbf{Data Sources}
            \begin{itemize}
                \item \textbf{BACI}: CEPII bilateral trade data
                \item \textbf{CEPII}: Gravity variables (distance, contiguity, etc.)
                \item \textbf{WDI}: World Development Indicators
            \end{itemize}
        \end{column}

        \begin{column}{0.5\textwidth}
            \textbf{Sample Construction}
            \begin{itemize}
                \item 78 ACP countries
                \item 7 Regional Economic Communities
                \item Time period: 1995–2020
            \end{itemize}
        \end{column}
    \end{columns}

    \vspace{0.3cm}

    \textbf{Key Variables}
    \begin{itemize}
        \item \textit{Dependent Variable}: Bilateral trade (EU-ACP), estimated via PPML
        \item \textit{Independent Variable}: Intra-regional trade share
        \item \textit{Controls}: Distance, common language, colonial ties, GDP
    \end{itemize}
\end{frame}

\begin{frame}{Descriptive Evidence}
    \begin{columns}
        \begin{column}{0.5\textwidth}
            \begin{block}{Figure 1a: Regional Trends}
                \centering
                \textit{Place Figure 1a Here}\\
                \texttt{it\_share\_time\_series.png}
            \end{block}
        \end{column}
        \begin{column}{0.5\textwidth}
            \begin{block}{Figure 1b: IT Share vs Trade}
                \centering
                \textit{Place Figure 1b Here}\\
                \texttt{stumbling\_block\_scatter.png}
            \end{block}
        \end{column}
    \end{columns}

    \vspace{0.3cm}
    \centering
    \small\textbf{Regional Integration and EU-ACP Trade: Descriptive Evidence}
\end{frame}

% ============================================
% EMPIRICAL STRATEGY
% ============================================
\section{EMPIRICAL STRATEGY}
\begin{frame}{Gravity Model Specification}
    \textbf{Model Equation}
    \begin{align}
        \ln(TRADE_{ijt}) &= \beta_0 + \beta_1 IT\_SHARE_{jt} + \beta_2 EPA_{jt} \\
        &\quad + \beta_3 (IT\_SHARE_{jt} \times EPA_{jt}) \\
        &\quad + \gamma X_{ijt} + \mu_{ij} + \theta_t + \epsilon_{ijt}
    \end{align}

    \vspace{0.2cm}

    \textbf{Key Variables}
    \begin{itemize}
        \item $IT\_SHARE_{jt}$: Intra-regional trade share for country $j$ at time $t$
        \item $EPA_{jt}$: Time-varying EPA treatment indicator
        \item $X_{ijt}$: Standard gravity controls
        \item $\mu_{ij}$: Country-pair fixed effects
        \item $\theta_t$: Time fixed effects
    \end{itemize}
\end{frame}

\begin{frame}{Identification Strategy}
    \textbf{Fixed Effects Structure}
    \begin{itemize}
        \item \textbf{Exporter $\times$ Year FE}: Control for time-varying exporter characteristics
        \item \textbf{ACP Importer FE}: Control for time-invariant importer characteristics
        \item \textbf{Year FE}: Control for common time shocks
    \end{itemize}

    \vspace{0.3cm}

    \textbf{Identification}
    \begin{itemize}
        \item Within-country variation in IT share over time
        \item Standard gravity controls: distance, language, colonial ties
    \end{itemize}
\end{frame}

% ============================================
% MAIN RESULTS
% ============================================
\section{MAIN RESULTS}
\begin{frame}{Main Results: Gravity Model Estimates}
    \begin{table}
        \centering
        \tiny
        \begin{tabular}{lccccc}
            \toprule
            & \makecell{Baseline\\PPML} & \makecell{IT Share\\PPML} & \makecell{IT $\times$ EPA\\PPML} &
\makecell{IT\\Intensity} & \makecell{IT Share\\OLS} \\
            \midrule
            ln(Distance) & -1.717$^{***}$ & -1.769$^{***}$ & -1.769$^{***}$ & -1.764$^{***}$ & -1.195$^{***}$ \\
            & (0.607) & (0.639) & (0.639) & (0.639) & (0.240) \\
            Common Language & 1.113$^{***}$ & 1.119$^{***}$ & 1.119$^{***}$ & 1.120$^{***}$ & 0.687$^{***}$ \\
            & (0.218) & (0.222) & (0.222) & (0.222) & (0.125) \\
            Colonial Tie & 0.205 & 0.186 & 0.186 & 0.184 & 1.251$^{***}$ \\
            & (0.228) & (0.232) & (0.232) & (0.232) & (0.202) \\
            EPA (=1 in force) & -0.031 & -0.058 & -0.158$^{**}$ & -0.055 & -0.136$^{***}$ \\
            & (0.047) & (0.044) & (0.066) & (0.044) & (0.044) \\
            Intra-REC Trade Share & & -1.755$^{***}$ & -1.758$^{***}$ & & -1.526$^{***}$ \\
            & & (0.384) & (0.378) & & (0.245) \\
            EPA $\times$ IT Share & & & 0.901$^{**}$ & & \\
            & & & (0.422) & & \\
            Intra-REC Trade Intensity & & & & -0.006$^{***}$ & \\
            & & & & (0.002) & \\
            \midrule
            Obs. & 44,054 & 43,238 & 43,238 & 43,238 & 40,334 \\
            R$^2$ & & & & & 0.755 \\
            \bottomrule
        \end{tabular}
    \end{table}
    \vspace{-0.3cm}
    \tiny\textit{Notes: } Clustered (pair\_id) SE in parentheses. $^{***}p<0.01$, $^{**}p<0.05$. Columns (1)-(4):
PPML. Column (5): OLS.
\end{frame}

\begin{frame}{Interpretation of Magnitudes}
    \textbf{Economic Interpretation}
    \begin{itemize}
        \item At mean intra-regional trade share (10$\%$): $\approx$16$\%$ lower EU-ACP bilateral trade
        \item At 90th percentile (26$\%$): $\approx$36$\%$ lower bilateral trade
    \end{itemize}

    \vspace{0.3cm}

    \textbf{EPA Offset Effect}
    \begin{itemize}
        \item EPAs partially mitigate the negative relationship
        \item Interaction term: $+0.90^{**}$ (p$<0.05$)
    \end{itemize}

    \vspace{0.3cm}

    \textbf{Trade Direction}
    \begin{itemize}
        \item Effect stronger for EU exports to ACP ($-1.86$)
        \item vs. ACP exports to EU ($-1.60$)
    \end{itemize}
\end{frame}

\begin{frame}{Heterogeneity: REC Effects}
    \begin{columns}
        \begin{column}{0.5\textwidth}
            \begin{block}{Figure 2a: REC Heterogeneity}
                \centering
                \textit{Place Figure Here}\\
                \texttt{rec\_coef\_plot.png}
            \end{block}
        \end{column}
        \begin{column}{0.5\textwidth}
            \begin{block}{Figure 2b: Marginal Effects}
                \centering
                \textit{Place Figure Here}\\
                \texttt{marginal\_effects\_it\_share.png}
            \end{block}
        \end{column}
    \end{columns}

    \vspace{0.2cm}
    \centering
    \small\textbf{Effect of Intra-REC Trade Share on EU-ACP Bilateral Trade}
\end{frame}

% ============================================
% MECHANISMS & HETEROGENEITY
% ============================================
\section{MECHANISMS AND HETEROGENEITY}
\begin{frame}{Heterogeneity Analysis}
    \textbf{REC-by-REC Subsamples}
    \begin{itemize}
        \item ECOWAS, SADC, EAC, CARICOM, CEMAC, COMESA, Pacific
        \item All show negative effects, but substantial variation
        \item PIF (Pacific Islands): largest negative coefficient ($-29.6$)
        \item Central Africa: smallest effect ($-0.81$, not significant)
    \end{itemize}

    \vspace{0.3cm}

    \textbf{Regional Subsamples}
    \begin{itemize}
        \item Africa: largest sample, significant effects
        \item Caribbean: moderate effects
        \item Pacific: limited data, imprecise estimates
    \end{itemize}
\end{frame}

% ============================================
% ROBUSTNESS CHECKS
% ============================================
\section{ROBUSTNESS CHECKS}
\begin{frame}{Robustness Checks}
    \begin{columns}
        \begin{column}{0.5\textwidth}
            \textbf{Sample Sensitivity}
            \begin{itemize}
                \item Excluding outlier countries
                \item Different time windows
                \item Alternative sample definitions
            \end{itemize}
        \end{column}

        \begin{column}{0.5\textwidth}
            \textbf{Specification Alternatives}
            \begin{itemize}
                \item Alternative fixed effects
                \item Different control variables
                \item Nonlinear specifications
            \end{itemize}
        \end{column}
    \end{columns}

    \vspace{0.3cm}

    \textbf{OLS vs. PPML}
    \begin{itemize}
        \item Poisson Pseudo-Maximum Likelihood addresses zero trade flows
        \item Results remain robust across specifications
    \end{itemize}
\end{frame}

% ============================================
% CONCLUSION
% ============================================
\section{CONCLUSION}
\begin{frame}{Conclusion}
    \textbf{Summary of Findings}
    \begin{enumerate}
        \item Strong evidence of ``stumbling block'' effect
        \item Higher intra-regional trade shares reduce EU-ACP bilateral trade
        \item Magnitude: 16$\%$ to 36$\%$ reduction depending on IT share level
        \item EPAs partially offset negative effects
    \end{enumerate}

    \vspace{0.3cm}

    \textbf{Policy Implications}
    \begin{itemize}
        \item Trade agreements must account for regional integration effects
        \item EPAs can mitigate trade-diverting effects of RECs
    \end{itemize}

    \vspace{0.3cm}

    \textbf{Limitations}
    \begin{itemize}
        \item Data constraints for smaller ACP countries
        \item Could extend to other developing regions
    \end{itemize}
\end{frame}

% ============================================
% THANK YOU SLIDE
% ============================================
\begin{frame}
    \begin{center}
        \Huge\textbf{Thank You!}

        \vspace{0.8cm}

        \Large Questions?

        \vspace{0.8cm}

        \normalsize
        \textit{Noah Damski} \\
        \texttt{noah.damski@ufl.edu}
    \end{center}
\end{frame}

\end{document}